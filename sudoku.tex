\documentclass{article}
\usepackage{bigpage}
\usepackage{blockpar}
\usepackage{blocktitle}
\usepackage{czt}
\usepackage{url}

\title{Sudoku}
\author{Bart Massey}
\date{29 March 2016}

\begin{document}
\maketitle

This document describes the format of a Sudoku solution.
It is assumed that the reader is at least familiar with the
Sudoku grid. We begin by describing some of the sizes and
dimensions relevant to the problem.

\begin{zed}
  SQUARE == 3 \\
  SIDE == SQUARE * SQUARE \\
  ROW == 1 \upto SIDE \\
  COL == 1 \upto SIDE \\
  VALUE == 1 \upto SIDE
\end{zed}

An important concept in Sudoku is the ``sub-board'' or
``group''.  It is important to know what coordinate group a
given coordinate is in.

\begin{axdef}
  group : 1 \upto SIDE \fun 1 \upto SQUARE
\where
  \forall x : 1 \upto SIDE @ \\
  \t1 group(x) = (x - 1) \div SQUARE + 1
\end{axdef}

There are basically three constraints on a partial Sudoku
solution: the same value cannot appear twice within a row;
the same value cannot appear twice within a column; the same
value cannot appear twice within a sub-board.

\begin{schema}{Sudoku}
  board : ROW \cross COL \pfun VALUE
\where
\forall r : ROW; c_1, c_2: COL | \\
  \t1 c_1 \neq c_2 \land \\
  \t1 (r, c_1) \in \dom(board) \land (r, c_2) \in \dom(board) @ \\
  \t2   board(r, c_1) \neq board(r, c_2)
\also
  \forall r_1, r_2 : ROW; c: COL | \\
  \t1 r_1 \neq r_2 \land \\
  \t1 (r_1, c) \in \dom(board) \land (r_2, c) \in \dom(board) @ \\
  \t2   board(r_1, c) \neq board(r_2, c)
\also
  \forall r_1, r_2 : ROW; c_1, c_2: COL | \\
  \t1 (r_1, c_1) \neq (r_2, c_2) \land \\
  \t1 (r_1, c_1) \in \dom(board) \land (r_2, c_2) \in \dom(board) \land \\
  \t1 group(r_1) = group(r_2) \land group(c_1) = group(c_2) @ \\
  \t2   board(r_1, c_1) \neq board(r_2, c_2)
\end{schema}

A solution to a Sudoku problem is a total assignment of
values to squares of the problem that respects the initial
partial assignment and the legality constraints.

\begin{schema}{SolveSudoku}
  problem? : Sudoku
\also
  solution! : Sudoku
\where
  problem?.board \subseteq solution!.board
\also
  solution!.board \in ROW \cross COL \fun VALUE
\end{schema}

\end{document}
