\documentclass{article}
\usepackage{bigpage}
\usepackage{blockpar}
\usepackage{blocktitle}
\usepackage{czt}
\usepackage{url}

\title{Solving Sudoku via SAT}
\author{Bart Massey}
\date{12 April 2016}

\begin{document}
\maketitle

In this document, we describe the construction of a Sudoku
solver that operates in three phases: construction of a prop
CNF SAT instance from the Sudoku instance, use of a SAT solver
to solve the SAT instance, and extracting the solution to the
Sudoku instance from the SAT solution. The constructor and
extractor are written in Python, and off-the-shelf SAT
solvers are used. Runtimes are very fast independent of
problem ``difficulty'', on the order of 150 milliseconds on
the author's desktop box.

\section{Solving Sudoku}

This section describes the properties of a Sudoku solution.
It is written using the Z specification notation for
precision and clarity.  It is assumed that the reader is at
least familiar with the Sudoku grid.

We begin by describing some of the sizes and dimensions
relevant to the problem.

\begin{zed}
  SQUARE == 3 \\
  SIDE == SQUARE * SQUARE \\
  ROW == 1 \upto SIDE \\
  COL == 1 \upto SIDE \\
  VALUE == 1 \upto SIDE
\end{zed}

An important concept in Sudoku is the ``sub-board'' or
``group''.  It is important to know what coordinate group a
given coordinate is in.

\begin{axdef}
  group : 1 \upto SIDE \fun 1 \upto SQUARE
\where
  \forall x : 1 \upto SIDE @ \\
  \t1 group(x) = (x - 1) \div SQUARE + 1
\end{axdef}

There are basically three constraints on a partial Sudoku
solution: the same value cannot appear twice within a row;
the same value cannot appear twice within a column; the same
value cannot appear twice within a sub-board.

\begin{schema}{Sudoku}
  board : ROW \cross COL \pfun VALUE
\where
\forall r : ROW; c_1, c_2: COL | \\
  \t1 c_1 \neq c_2 \land \\
  \t1 (r, c_1) \in \dom(board) \land (r, c_2) \in \dom(board) @ \\
  \t2   board(r, c_1) \neq board(r, c_2)
\also
  \forall r_1, r_2 : ROW; c: COL | \\
  \t1 r_1 \neq r_2 \land \\
  \t1 (r_1, c) \in \dom(board) \land (r_2, c) \in \dom(board) @ \\
  \t2   board(r_1, c) \neq board(r_2, c)
\also
  \forall r_1, r_2 : ROW; c_1, c_2: COL | \\
  \t1 (r_1, c_1) \neq (r_2, c_2) \land \\
  \t1 (r_1, c_1) \in \dom(board) \land (r_2, c_2) \in \dom(board) \land \\
  \t1 group(r_1) = group(r_2) \land group(c_1) = group(c_2) @ \\
  \t2   board(r_1, c_1) \neq board(r_2, c_2)
\end{schema}

A solution to a Sudoku instance is a total assignment of
values to squares of the instance that respects the initial
partial assignment and the legality constraints.

\begin{schema}{SolveSudoku}
  problem? : Sudoku
\also
  solution! : Sudoku
\where
  problem?.board \subseteq solution!.board
\also
  solution!.board \in ROW \cross COL \fun VALUE
\end{schema}

\section{Prop. CNF SAT Instance Extraction}

The logical description of the previous section greatly
facilitates reducing a Sudoku instance to a Prop. CNF SAT
instance whose solution gives a solution to the Sudoku
instance.

We will represent the board relation using atoms of the form
$B_{rcv}$ which will be interpreted as true iff the board at
row $r$ and column $c$ has value $v$. There are $9^3=729$
such atoms. We will establish a series of constraints,
extracted from the Z specification, that together ensure
that any satisfying assignment to the $B$s will be
interpretable as a solution to the Sudoku instance.

\begin{enumerate}
\item We require that the
  solution match the given Sudoku instance. To do this,
  we emit unary clauses of the form $$
    B_{rcv}
  $$ for each $r$, $c$ and $v$ specified in the Sudoku
    instance description.
\item We require that the $board$ relation
  is total. To do this, we emit clauses of the form $$
    (B_{rc1} \lor B_{rc2} \lor \ldots \lor B_{rc9})
  $$ for every $r$ and $c$. Note that these are the
  only clauses in the description of arity greater than two.
\item We require that the $board$ relation
  is a function. To do this, we emit clauses of the form $$
    (\lnot B_{rcv_1} \lor \lnot B_{rcv_2})
  $$ for every $r$, $c$, $v_1$ and $v_2$ such that $v_1 \neq v_2$.
\item We require that no row contains the same value
  in two different columns. To do this, we emit clauses of
  the form $$
    (\lnot B_{rc_1v} \lor \lnot B_{rc_2v})
  $$ for every $r$, $c_1$, $c_2$ and $v$ such that $c_1 \neq c_2$.
\item We require that no column contains the same value
  in two different rows. To do this, we emit clauses of
  the form $$
    (\lnot B_{r_1cv} \lor \lnot B_{r_2cv})
  $$ for every $r_1$, $r_2$, $c$ and $v$ such that $r_1 \neq r_2$.
\item We require that the same value not appear in two
  different positions in the same sub-board. To do this, we
  emit clauses of the form $$
    (\lnot B_{r_1c_1v} \lor \lnot B_{r_2c_2v})
  $$ for every $r_1$, $r_2$, $c_1$, $c_2$ and $v$ such that
  $(r_1, c_1) \neq (r_2, c_2)$ but both coordinates are in
  the same sub-board.
\end{enumerate}

\section{Decoding The Answer}

Once a satsifying assignment is found for the problem of the
previous section, it remains to turn this assignment back
into a solved Sudoku board. To do so, we note that, since
$board$ is a total function, for each row $r$ and column
$c$ there will be exactly one value $v$ for which $B_{rcv}$
is true: this $v$ is the value which should be filled in
at position $(r, c)$ in the board.

\section{Implementation and Evaluation}

All of this is implemented using two Python programs, one to
encode and one to extract the answer, together with an
off-the-shelf SAT solver.

The two SAT solvers I have tried are the readily-available
open-source solvers Picosat and Minisat2. These solver both
require the same input format: the famous DIMACS format. In
this format, one row of text corresponds to one clause. Each
atom is given a number starting at 1. Positive literals are
denoted by atom number, negative literals by its integer
negation. The DIMACS output format is similar.

The running system has been tested on several Sudoku
instances, including ``the hardest Sudoku'' and another
``very hard'' instance. End-to-end run times are uniformly
in the 200 millisecond range. The program has also been
tested on a blank Sudoku board, generating a filled grid,
and on an unsolvable Sudoku instance, which it correctly
repots as unsolvable.

\end{document}
